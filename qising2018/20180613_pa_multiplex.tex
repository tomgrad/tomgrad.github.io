\documentclass[11pt]{article}  


\usepackage{graphicx}
 
%\renewcommand{\thesection}{\arabic{section}.}  
%\renewcommand{\thesubsection}{\thesection\arabic{subsection}.}  
\textwidth=15.0cm  
\textheight=23.0cm  
\topmargin=-50pt  
\oddsidemargin=0.5in  
  
  
\begin{document}  
  
\title{{\bf Pair approximation for multiplex networks}}  
\author{AA\\  
Faculty of Physics, Warsaw University of Technology, \\
Koszykowa 75, PL-00-662 Warsaw, Poland}  
  
\baselineskip=4ex   
%\begin{titlepage}  
\maketitle  
\begin{abstract}  
AA
\end{abstract}  
  
PACS numbers: 

Key words: 
  

\section{The model}

Nonequilibrium models for the opinion formation, the $q$-voter model and the $q$-Ising model are considered on
multiplex networks (MNs) with two layers denoted as $G^{(A)}$, $G^{(B)}$. 
The MNs under study consist of a fixed set of nodes $i=1,2,\ldots N$
at which two-state spins $s_{i}=\pm 1$ are placed, and two separetely generated layersof edges connecting the 
nodes which are complex networks, e.g., ERGs or RRGs. Only the case of fully overlapping MNs is considered.
Each node has degree $k^{(A)}$ ($k^{(B)}$) within the layer $G^{(A)}$ ($G^{(B)}$),
the mean degrees of nodes within the layers are $\langle k^{(A)}\rangle$ ($\langle k^{(B)}\rangle$)
and the joint degree distribution is $P\left( k^{(A)}, k^{(B)}\right)$.

For the analytic study of the model PA is used.
Within this approach the macroscopic quantities characterizing the model are concentrations of spins with orientation
up and of active bonds connecting nodes occupied by spins with opposite orientations. In the case of a model
on a MN a single quantity $c$ is enough to characterize the concentration of spins with orientation up 
(normalized to the number of nodes $N$), but two quantities $b^{(A)}$, $b^{(B)}$ 
characterizing the concentrations of active bonds within different layers 
(normalized to the total numbers of edges $N\langle k^{(A)}\rangle/2$ ($N\langle k^{(B)}\rangle/2$)
within the layer$G^{(A)}$ ($G^{(B)}$)) are necessary. The flipping probability of a spin located in a node
charcterized by the degrees $k^{(A)}$, $k^{(B)}$ if the number of active edges attached to it within the layer 
$G^{(A)}$ ($G^{(B)}$) is $i^{(A)}$ ($i^{(B)}$) is denoted by
$f_{k^{(A)},k^{(B)}}\left( i^{(A)},i^{(B)},q,\hat{p}\right)$,
where $\hat{p}$ is a parameter characterizing the degree of stochasticity in a model
(parameter $p$ characterizing independence in the case of the $q$-voter model and temperatute $T$ in the
case of the $q$-neighbor Ising model).

The rate equations for the macroscopic concentrations $c$, $b^{(A)}$, $b^{(B)}$ can be derived as in Ref.\ 
\cite{Jedrzejewski17} taking into account that each spin flip 
in a node with degrees $k^{(A)}$, $k^{(B)}$ and the numbers of active edges attached to it $i^{(A)}$, $i^{(B)}$ 
affects simultaneously concentrations of active edges within both layers changing them by
\begin{eqnarray}
\Delta_{b^{(A)}}&=&\frac{2}{N\langle k^{(A)}\rangle}\left( k^{(A)}-2i^{(A)} \right) \nonumber\\
\Delta_{b^{(B)}}&=&\frac{2}{N\langle k^{(B)}\rangle}\left( k^{(B)}-2i^{(B)} \right).
\end{eqnarray}
The probabilities that in each MCSS first the spin with orientation $j\in \left\{ \uparrow,\downarrow\right\}$ 
is selected and then its neighbor within the layer $G^{(A)}$ ($G^{(B)}$) with opposite orientation is selected are denoted
by $\theta_{j}^{(A)}$ ($\theta_{j}^{(B)}$),
\begin{equation}
\theta_{\uparrow}^{(A)}=\frac{b^{(A)}}{2c},\;\; \theta_{\downarrow}^{(A)}=\frac{b^{(A)}}{2(1-c)}
\end{equation}
\begin{equation}
\theta_{\uparrow}^{(B)}=\frac{b^{(B)}}{2c},\;\; \theta_{\downarrow}^{(B)}=\frac{b^{(B)}}{2(1-c)}
\end{equation}

\section{General rate equations}

General rate equations for the concentrations $c$, $b^{(A)}$, $b^{(B)}$ are
\begin{eqnarray}
\frac{\partial c}{\partial t} &=&\gamma^{+}-\gamma^{-}, 
\label{rate1}\\
\frac{\partial b^{(A)}}{\partial t} &=& \frac{2}{\langle k^{(A)}\rangle} \sum_{j\in \left\{ \uparrow,\downarrow\right\}} c_{j}
\sum_{k^{(A)},k^{(B)}} P\left( k^{(A)}, k^{(B)}\right) 
\nonumber\\
&\times&
\sum_{i^{(A)}=0}^{k^{(A)}} \sum_{i^{(B)}=0}^{k^{(B)}} 
{k^{(A)} \choose i^{(A)} } \theta_{j}^{(A)i^{(A)}}\left( 1-\theta_{j}^{(A)}\right)^{k^{(A)}-i^{(A)}}
{k^{(B)} \choose i^{(B)} } \theta_{j}^{(B)i^{(B)}}\left( 1-\theta_{j}^{(B)}\right)^{k^{(B)}-i^{(B)}}
\nonumber\\
&\times& f_{k^{(A)},k^{(B)}}\left( i^{(A)},i^{(B)},q,\hat{p}\right) \left( k^{(A)}-2i^{(A)}\right), \label{rate2}
\end{eqnarray}
and a complementary equation for $b^{(B)}$ which can be obtained from Eq.\ (\ref{rate2}) by exchanging the
superscripts $A$, $B$.

The final form of the above system of equations depends on the joint degree distribution 
$P\left( k^{(A)}, k^{(B)}\right) $ and 
the spin update rule determining the form of the spin flip rate 
$f_{k^{(A)},k^{(B)}}\left( i^{(A)},i^{(B)},q,\hat{p}\right)$. 
Concerning the former problem, particularly simple situation occurs if the layers 
of the MN are separately and independently generated networks
with degree disributions $P\left( k^{(A)}\right)$, $P\left( k^{(B)}\right)$ which yields the joint degree distribution
$P\left( k^{(A)}, k^{(B)}\right)= P\left( k^{(A)}\right) P\left( k^{(B)}\right)$. 
Concerning the latter problem, particularly simple situation occurs for the LOCAL AND update rule
defined in Ref.\ \cite{Chmiel15}, where the probability of each spin flip is a product of probabilities characteristic of a given
model evaluated separately for each layer and thus
\begin{displaymath}
f_{k^{(A)},k^{(B)}}\left( i^{(A)},i^{(B)},q,\hat{p}\right)=
f_{k^{(A)}}\left( i^{(A)},q,\hat{p}\right) f_{k^{(B)}}\left( i^{(B)},q,\hat{p}\right).
\end{displaymath}
Under the two above-mentioned assumptions 
the summations over $k^{(A)}$, $k^{(B)}$ as well as over
$i^{(A)}$, $i^{(B)}$ in Eq.\ (\ref{rate2}) can be performed separately which yields
\begin{eqnarray}
\frac{\partial b^{(A)}}{\partial t} &=& \frac{2}{\langle k^{(A)}\rangle} \sum_{j\in \left\{ \uparrow,\downarrow\right\}} c_{j}
\nonumber\\
&\times&
\sum_{k^{(A)}} P\left( k^{(A)}\right)  
\sum_{i^{(A)}=0}^{k^{(A)}}
{k^{(A)} \choose i^{(A)} } \theta_{j}^{(A)i^{(A)}}\left( 1-\theta_{j}^{(A)}\right)^{k^{(A)}-i^{(A)}}
f_{k^{(A)}}\left( i^{(A)},q,\hat{p}\right) \left( k^{(A)}-2i^{(A)}\right) \nonumber\\
&\times& 
\sum_{k^{(B)}} P\left( k^{(B)}\right)  
\sum_{i^{(B)}=0}^{k^{(B)}}
{k^{(B)} \choose i^{(B)} } \theta_{j}^{(B)i^{(B)}}\left( 1-\theta_{j}^{(B)}\right)^{k^{(B)}-i^{(B)}}
f_{k^{(B)}}\left( i^{(B)},q,\hat{p}\right),
\label{rate3}
\end{eqnarray}
and a complementary equation for $b^{(B)}$ which can be obtained from Eq.\ (\ref{rate3}) by exchanging the
superscripts $A$, $B$.

\section{Rate equations for the $q$-voter model}

\subsection{LOCAL AND update rule}

For the $q$-voter model the spin flip rate is
\begin{eqnarray}
f_{k^{(A)}}\left( i^{(A)},q,p\right) &=&
(1-p) \frac{\prod_{j=1}^{q}\left( i^{(A)}-j+1\right)}{\prod_{j=1}^{q}\left( k^{(A)}-j+1\right)} +\frac{p}{2}
\nonumber\\
&=&
(1-p)\frac{i^{(A)}!\left( k^{(A)} -q\right)!}{k^{(A)}!\left(i^{(A)}-q\right)!} +\frac{p}{2}.
\label{fkv}
\end{eqnarray}
 The formula for $f_{k^{(B)}}\left( i^{(B)},q,p\right)$ 
can be obtained from Eq.\ (\ref{fk}) by changing the superscript $A$ into $B$. Similarly, the rates
$\gamma^{+}$, $\gamma^{-}$ are
\begin{eqnarray}
\gamma^{+}&=& (1-c) \left[ (1-p)\theta_{\downarrow}^{(A)q} +\frac{p}{2}\right]
\left[ (1-p)\theta_{\downarrow}^{(B)q} +\frac{p}{2}\right], 
\label{gammaplusv}
\\
\gamma^{-}&=& c \left[ (1-p)\theta_{\uparrow}^{(A)q} +\frac{p}{2}\right]
\left[ (1-p)\theta_{\uparrow}^{(B)q} +\frac{p}{2}\right].
\label{gammaminusv}
\end{eqnarray}

Substituting Eq.\ (\ref{fkv}) in Eq.\ (\ref{rate3}) and performing summations as in Ref.\ \cite{Jedrzejewski17} as well as
substituting Eq.\ (\ref{gammaplusv},\ref{gammaminusv}) in Eq.\ (\ref{rate1}) yields the following system of rate
equations for the macroscopic quantities $c$, $b^{(A)}$, $b^{(B)}$,
\begin{eqnarray}
\frac{\partial c}{\partial t} &=&\gamma^{+}-\gamma^{-}, 
\label{systemcb3v}\\
\frac{\partial b^{(A)}}{\partial t} &=& \frac{2}{\langle k^{(A)}\rangle} \sum_{j\in \left\{ \uparrow,\downarrow\right\}} c_{j} \nonumber\\
& \times&  \left\{ (1-p) \theta_{j}^{(A)q}
\left[  \langle k^{(A)}\rangle-2q -2 \left( \langle k^{(A)}\rangle -q\right) \theta_{j}^{(A)} \right]
+\frac{p}{2}\langle k^{(A)}\rangle \left(1-2 \theta_{j}^{(A)}\right) \right\} \nonumber\\
&\times& \left[ (1-p) \theta_{j}^{(B)q} +\frac{p}{2}\right],
\label{systemcb4v}
\end{eqnarray}
and the complementary equation for $b^{(B)}$ which can be obtained from Eq.\ (\ref{systemcb4v}) by exchanging the 
superscripts $A$ and $B$.

Further simplification of the above system of equations
can be achieved if the two layers are independently and separately generated
networks with identical degree distributions $P\left( k^{(A)}\right) =P\left( k^{(B)}\right) = P(k)$, 
e.g., two ERGs or RRGs with the same mean degrees of nodes
$\langle k^{(A)} \rangle = \langle k^{(B)}\rangle = \langle k\rangle$. Due to the symmetry of Eq.\ (\ref{systemcb4v}) and
the complementary equation for $b^{(B)}$ a solution exists with equal active bond concentrations within both layers,
$b^{(A)}=b^{(B)} =b$ (and thus with $\theta_{j}^{(A)} =\theta_{j}^{(B)} =\theta_{j}$, 
$j\in \left\{ \uparrow,\downarrow\right\}$). 
Then the two rate equations for the active bond concentrations can be replaced with a single equation for $b$.
Hence, in this case the model on a MN is described by only two macroscopic 
quantities $c$, $b$ obeying the equations
\begin{eqnarray}
\frac{\partial c}{\partial t} &=&(1-c) \left[ (1-p)\theta_{\downarrow}^{q} +\frac{p}{2}\right]^2 -
c\left[ (1-p)\theta_{\uparrow}^{q} +\frac{p}{2}\right]^2 \\
\frac{\partial b}{\partial t} &=& \frac{2}{\langle k\rangle} \sum_{j\in \left\{ \uparrow,\downarrow\right\}} c_{j} \nonumber\\
& \times&  
 \left\{ (1-p) \theta_{j}^{q}
\left[  \langle k\rangle-2q -2 \left( \langle k\rangle -q\right) \theta_{j} \right]
+\frac{p}{2}\langle k\rangle \left(1-2 \theta_{j}\right) \right\}
 \left[ (1-p) \theta_{j}^{q} +\frac{p}{2}\right].
\end{eqnarray}
In particular, for $p=0$ the above system of equations reduces to
\begin{eqnarray}
\frac{\partial c}{\partial t} &=&(1-c) \theta_{\downarrow}^{2q} - c \theta_{\uparrow}^{2q}\\
\frac{\partial b}{\partial t} &=& \frac{2}{\langle k\rangle} \sum_{j\in \left\{ \uparrow,\downarrow\right\}} c_{j} 
\theta_{j}^{2q}
\left[  \langle k\rangle-2q -2 \left( \langle k\rangle -q\right) \theta_{j} \right],
\end{eqnarray}
which is equivalent to the equations for the macroscopic concentrations of spins with orientation up and of active bonds
obtained in the PA for the $2q$-voter model with $p=0$ on a network with the mean degree of nodes $2 \langle k\rangle$.
Thus, in the framework of the PA, the $q$-voter model with $p=0$ on a MN with two layers is equivalent to 
the $2q$-voter model on a network being a superposition of the two layers. However, this is not the case for $0<p \le 1$.

\subsection{GLOBAL AND update rule}

In this case the spin-flip rate cannot be written as a product of the rates evaluated separately for each layer. It takes a form
\begin{eqnarray}
f_{k^{(A)},k^{(B)}}\left( i^{(A)},i^{(B)},q,\hat{p}\right) &=&
(1-p) \frac{\prod_{j=1}^{q}\left( i^{(A)}-j+1\right)}{\prod_{j=1}^{q}\left( k^{(A)}-j+1\right)} 
\frac{\prod_{j^{\prime}=1}^{q}\left( i^{(B)}-j^{\prime}+1\right)}{\prod_{j^{\prime}=1}^{q}\left( k^{(B)}-j^{\prime}+1\right)} 
+\frac{p}{2}
\nonumber\\
&=&
(1-p)\frac{i^{(A)}!\left( k^{(A)} -q\right)!}{k^{(A)}!\left(i^{(A)}-q\right)!} 
\frac{i^{(B)}!\left( k^{(B)} -q\right)!}{k^{(B)}!\left(i^{(B)}-q\right)!}
+\frac{p}{2}.
\label{fkvG}
\end{eqnarray}
Similarly, the rates
$\gamma^{+}$, $\gamma^{-}$ are
\begin{eqnarray}
\gamma^{+}&=& (1-c) \left[ (1-p)\theta_{\downarrow}^{(A)q}\theta_{\downarrow}^{(B)q} +\frac{p}{2}\right], 
\label{gammaplusvG}
\\
\gamma^{-}&=& c \left[ (1-p)\theta_{\uparrow}^{(A)q} \theta_{\uparrow}^{(B)q}+\frac{p}{2}\right].
\label{gammaminusvG}
\end{eqnarray}
Substituting Eq.\ (\ref{fkvG}) in Eq.\ (\ref{rate2}) and performing summations as in Ref.\ \cite{Jedrzejewski17} as well as
substituting Eq.\ (\ref{gammaplusvG},\ref{gammaminusvG}) in Eq.\ (\ref{rate1}) yields the following system of rate
equations for the macroscopic quantities $c$, $b^{(A)}$, $b^{(B)}$,
\begin{eqnarray}
\frac{\partial c}{\partial t} &=&\gamma^{+}-\gamma^{-}, 
\label{systemcb3vG}\\
\frac{\partial b^{(A)}}{\partial t} &=& \frac{2}{\langle k^{(A)}\rangle} \sum_{j\in \left\{ \uparrow,\downarrow\right\}} c_{j} \nonumber\\
& \times&  \left\{ (1-p) \theta_{j}^{(A)q} \theta_{j}^{(B)q}
\left[  \langle k^{(A)}\rangle-2q -2 \left( \langle k^{(A)}\rangle -q\right) \theta_{j}^{(A)} \right]
+\frac{p}{2}\langle k^{(A)}\rangle \left(1-2 \theta_{j}^{(A)}\right) \right\}, \nonumber\\
&&
\label{systemcb4vG}
\end{eqnarray}
and the complementary equation for $b^{(B)}$ which can be obtained from Eq.\ (\ref{systemcb4v}) by exchanging the 
superscripts $A$ and $B$.

In the case of the two independently and separately generated layers with identical degree distributions $P\left( k^{(A)}\right) =P\left( k^{(B)}\right) = P(k)$
(thus with $\langle k^{(A)} \rangle = \langle k^{(B)}\rangle = \langle k\rangle$) the solution with $b^{(A)}=b^{(B)} =b$  obeys the system of equations
\begin{eqnarray}
\frac{\partial c}{\partial t} &=&(1-c) \left[ (1-p)\theta_{\downarrow}^{2q} +\frac{p}{2}\right] -
c\left[ (1-p)\theta_{\uparrow}^{2q} +\frac{p}{2}\right] \\
\frac{\partial b}{\partial t} &=& \frac{2}{\langle k\rangle} \sum_{j\in \left\{ \uparrow,\downarrow\right\}} c_{j} \nonumber\\
& \times&  
 \left\{ (1-p) \theta_{j}^{2q}
\left[  \langle k\rangle-2q -2 \left( \langle k\rangle -q\right) \theta_{j} \right]
+\frac{p}{2}\langle k\rangle \left(1-2 \theta_{j}\right) \right\}.
\end{eqnarray}
It can be easily verified that in the framework of the PA the above $q$-voter model with the GLOBAL AND update rule is equivalent to 
the $2q$-voter model on a network with the mean degree of nodes $2 \langle k\rangle$.
in a whole range of $p$, $0\le p\le1$.

\section{Rate equations for the $q$-neighbor Ising model}

For the $q$-neighbor Ising model the spin flip rate is 
\begin{eqnarray}
f_{k^{(A)}}\left( i^{(A)},q,T\right) &=& \frac{1}{{k^{(A)} \choose q}} \sum_{l=0}^{q} {i^{(A)} \choose l} 
{k^{(A)}-i^{(A)} \choose q-l} \min \left(1, e^{ -2\beta J(q-2l)}\right)
\nonumber\\
&=&  \frac{1}{{k^{(A)} \choose i}} \sum_{l=0}^{q} {k^{(A)}-q \choose i^{(A)}-l} {q \choose l}
\min \left(1, e^{ -2\beta J(q-2l)}\right),
\label{fk}
\end{eqnarray}
where, however, it should be remembered that the summation is performed only over such $l$ that additional constraints
$l\le i$ and $q-l \le k-i$ are fulfilled; these constraints can be formally included in Eq.\  (\ref{fk}) by assuming that 
${n \choose m}\equiv 0$ for $m<0$ and $n<m$. The formula for $f_{k^{(B)}}\left( i^{(B)},q,T\right)$ 
can be obtained from Eq.\ (\ref{fk}) by changing the superscript $A$ into $B$. Similarly, the rates
$\gamma^{+}$, $\gamma^{-}$ are
\begin{eqnarray}
\gamma^{+} &=& (1-c) 
\left[ \sum_{l=0}^{q} {q\choose l} \theta^{(A)l}_{\downarrow} \left( 1-\theta^{(A)}_{\downarrow} \right)^{q-l}
 \min \left(1, e^{ -2\beta J(q-2l)}\right) \right], \nonumber\\
&\times& 
\left[ \sum_{l^{\prime}=0}^{q} {q\choose l} \theta^{(B)l^{\prime}}_{\downarrow} 
\left( 1-\theta^{(B)}_{\downarrow} \right)^{q-l^{\prime}}
 \min \left(1, e^{ -2\beta J(q-2l^{\prime})}\right) \right], \label{gammaplus}\\
\gamma^{-} &=& c\left[ \sum_{l=0}^{q} {q\choose l} \theta_{\uparrow}^{(A)l} \left( 1-\theta_{\uparrow}^{(A)} \right)^{q-l}
 \min \left(1, e^{ -2\beta J(q-2l)}\right) \right] \nonumber\\
&\times&
\left[ \sum_{l^{\prime}=0}^{q} {q\choose l^{\prime}} \theta_{\uparrow}^{(B)l^{\prime}} 
\left( 1-\theta_{\uparrow}^{(B)} \right)^{q-l^{\prime}}
 \min \left(1, e^{ -2\beta J(q-2l^{\prime})}\right) \right]. \label{gammaminus}
\end{eqnarray}

Substituting Eq.\ (\ref{fk}) in Eq.\ (\ref{rate3}) and performing summations as in Ref.\ \cite{Chmiel17a} as well as
substituting Eq.\ (\ref{gammaplus},\ref{gammaminus}) in Eq.\ (\ref{rate1}) yields the following system of rate
equations for the macroscopic quantities $c$, $b^{(A)}$, $b^{(B)}$,
\begin{eqnarray}
\frac{\partial c}{\partial t} &=&\gamma^{+}-\gamma^{-}, 
\label{systemcb3}\\
\frac{\partial b^{(A)}}{\partial t} &=& \frac{2}{\langle k^{(A)}\rangle} \sum_{j\in \left\{ \uparrow,\downarrow\right\}} c_{j} \nonumber\\
& \times&  \left\{ \sum_{l=0}^{q} {q \choose l } 
\theta_{j}^{(A)l}\left( 1-\theta_{j}^{(A)}\right)^{q-l}
\left[  \langle k^{(A)}\rangle -2 \left( \langle k^{(A)}\rangle -q\right) \theta_{j}^{(A)} -2l\right] \right. \nonumber\\
&& \left. \times \min \left(1, e^{ -2\beta J(q-2l)}\right)\right\} \nonumber\\
&\times& \left\{  \sum_{l^{\prime}=0}^{q} {q \choose l^{\prime}} 
\theta_{j}^{(B)l^{\prime}}\left( 1-\theta_{j}^{(B)}\right)^{q-l^{\prime}}
\min \left(1, e^{ -2\beta J(q-2l^{\prime})}\right)
\right\},
\label{systemcb4}
\end{eqnarray}
and the complementary equation for $b^{(B)}$ which can be obtained from Eq.\ (\ref{systemcb4}) by exchanging the 
superscripts $A$ and $B$.

Again, in the case of two separately and independently generated layers with identical degree distributions the
above system of equations can be simplified to a system of only two equations for the macroscopic 
concentrations $c$, $b$. This can be achieved simply by dropping the superscripts $(A)$, $(B)$ in 
Eq.\ (\ref{gammaplus},\ref{gammaminus},\ref{systemcb3},\ref{systemcb4}).

\newpage

\begin{thebibliography}{99}

\bibitem{Jedrzejewski17}
A.\ J\c{e}drzejewski,
{\it Phys.\ Rev.\ E} {\bf 95}, 012307 (2017).

\bibitem{Chmiel15}
A.\ Chmiel and K.\ Sznajd-Weron,
{\it Phys.\ Rev.\ E} {\bf 92},  052812 (2015).

\bibitem{Chmiel17}
A.\ Chmiel, J.\ Sienkiewicz, K.\ Sznajd-Weron,
arxiv: 1611.07938

\bibitem{Chmiel17a}
A.\ Chmiel, T.\ Gradowski, A.\ Krawiecki,
{\it Int.\ J.\ Modern Phys.\ C}, submitted

\end{thebibliography}


\end{document}  

